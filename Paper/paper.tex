%
% LaTeX template for prepartion of submissions to PLDI'15
%
% Requires temporary version of sigplanconf style file provided on
% PLDI'15 web site.
% 
\documentclass[pldi]{sigplanconf-pldi15}

%
% the following standard packages may be helpful, but are not required
%
\usepackage{SIunits}            % typset units correctly
\usepackage{courier}            % standard fixed width font
\usepackage[scaled]{helvet} % see www.ctan.org/get/macros/latex/required/psnfss/psnfss2e.pdf
\usepackage{url}                  % format URLs
\usepackage{listings}          % format code
\usepackage{enumitem}      % adjust spacing in enums
\usepackage[colorlinks=true,allcolors=blue,breaklinks,draft=false]{hyperref}   % hyperlinks, including DOIs and URLs in bibliography

\begin{document}

\title{GastroGnome: A Language for Recipes}
\authorinfo{William Clarkson}{}{william.clarkson@tufts.edu}
\authorinfo{Andrew Mendelsohn}{}{andrew.mendelsohn@tufts.edu}

\maketitle
\begin{abstract}
The majority of recipes are written in natural language, and there exists no
simple, standardized, and unambiguous representation of recipes which lends
itself to consumption by a computer program. With this in mind, we have
developed a new programming language for the specification of recipes. Our
primary goals to provide a readable and expressive language which can serve as
a foundation for a myriad of future tools which perform computations over
recipes.
\end{abstract}

\section{Introduction}
Recent years have seen increased application of computational techniques to a
variety of domains. Great advances have been made in fields such as
computational biology. However, while some domains operate over
well-structured data (e.g. genomic data), some do not lend themselves so
readily to computation. We believe that the culinary arts is one such domain.
A structured language for the unambiguous description of recipes would open
the domain of cooking to the application of computational techniques for the
analysis of recipes and the generation of useful artifacts based on such
recipes. In Section \ref{sec:exampleusage}, we will discuss some tools that we
have created to demonstrate the language’s promise as a foundation for
interesting future computation over recipes.

At a high level, we decided to create a language with a syntax that feels
declarative, since recipes often contain no logic, and a more imperative
structure would likely be less readable. We aimed to take cues from the
functional paradigm, and represent recipes as repeated composition of
increasingly complex parts, starting with atomic data.

\section{Design}
We settled on a simple syntax for the language, with a handful of declaration
forms and expression forms, for which we provide some examples in Figures
\ref{fig:dec} and \ref{fig:exp}. A named recipe is introduced with an ingredient declaration, and its right-hand side, so to speak, is composed of an ingredient,
which can be a single atomic ingredient quantity expression, or an action which
combines a list of ingredients. Quantity declarations enable the default
quantity for an ingredient to be specified at top level so the names can later
be referred to without a quantity, as is common practice for many cookbooks.
Action declarations allow the user to define a new action in terms of another
action in conjunction with adverbs to specify it more precisely. In natural
language terms, our grammar for recipes consists of ways of introducing
ingredients (nouns), and ways of combining ingredients (verbs with optional
adverbs).

While the compositionality of our language makes it reminiscent of Lisp
S-expressions, we decided to use required indentation instead of matched
delimiters like parentheses to structure our code, as we thought this would
lead to a more readable program, and one which would be less daunting to a
cook with minimal programming experience.

A full recipe expressed in GastroGnome is shown in Figure \ref{fig:waffles}.
Several ingredient declarations (e.g. \texttt{Dry} \texttt{Ingredients},
\texttt{Batter}, etc) are used.  The right-hand side ingredient expression for
\texttt{Batter} refers to \texttt{Dry} \texttt{Ingredients}, which is defined
before, demonstrating the use of variable binding. Additionally, the
ingredients in \texttt{Batter} are defined without explicit quantities, so the
quantities declared at the top of the recipe will be used, which illustrates
the use of quantity declarations.  Additionally, an action declaration is used
to declare the new verb \texttt{INCORPORATE}, in terms of another verb,
\texttt{BEAT}, which is perhaps more familiar to the chef.

\begin{figure}
	\begin{tabular}{ l l }
		Ingredient Quantity & \tt 3 cup Flour \rm or \tt 4 Egg \\
		Name & \tt Flour \\
		Action & \tt BLEND \rm or \tt MIX until "smooth" \\
		Quantity & \tt 4 tsp \rm or \tt 2 \\
		Unit & \tt cup \rm or \tt tbsp
	\end{tabular}
	\caption{GastroGnome declaration forms}
	\label{fig:dec}
\end{figure}	
\begin{figure}	
	\begin{tabular}{ l l }
		Ingredient Declaration & \tt Mixture: $\hookleftarrow$ 1 cup Water \\ 
		Quantity Declaration & \tt 3 cup Flour \\
		Action Declaration & \tt WHISK = STIR with "whisk" \\
		Unit Declaration & \tt 1 gallon = 4 quart
	\end{tabular}
	\caption{GastroGnome expression forms}
	\label{fig:exp}
\end{figure}

\begin{figure}
	\begin{verbatim}
	2 Egg
	7/4 cup Milk
	1/2 cup Vegetable Oil
	1/2 tsp Vanilla

	INCORPORATE = BEAT until "smooth"

	Dry Ingredients:
	  MIX
	    2 cup Flour
	    1 tablespoon Sugar
	    4 teaspoon Baking Powder
	    1/4 tsp Salt
 
	Batter:
	  INCORPORATE in "large bowl" 
	    BEAT until "fluffy"
	      Egg
	    Dry Ingredients
	    Milk
	    Vegetable Oil
	    Vanilla

	Prepared Waffle Iron:
	  SPRAY with "Cooking Spray"
	    PREHEAT for "15 min"
	      Waffle Iron
 
	Waffles:
	  COOK at "300 F"
	  	   in "Prepared Waffle Iron"
	  	   until "golden brown"
	    Batter	
	\end{verbatim}
	\caption{Recipe for waffles expressed in GastroGnome}
	\label{fig:waffles}
\end{figure}

\section{Implementation}
Our language is parsed using the Parsec parser combinator library, which
enabled it to be implemented fairly straightforwardly and in an organized and
readable manner which will allow it to be extended easily in the future if
additional language constructs are added to the syntax.

GastroGnome is embedded in Haskell using the quasiquoting facility provided by
Template Haskell. We have implemented the declaration form, which allows a
quasiquotation at top level which contains any number of top-level GastroGnome
declarations. This is the primary manner in which a recipe or set of recipes
are introduced. Additionally, we have implemented the expression form, which
allows a single quoted ingredient expression to be introduced to Haskell. This
enables some simplifying shorthand in the programmatic construction of recipes
over manual use of the exposed data constructors for the GastroGnome
representation of recipes.

When GastroGnome syntax is quasiquoted in Haskell, it is parsed, using the
appropriate parser for either full programs (list of GastroGnome declarations)
or single ingredient expressions, into an algebraic data structure
representing the abstract syntax. This then undergoes a process which we have
termed “reification” due to it’s passing similarity to the process described
in the Mainland quasiquoting paper \cite{mainland07quasiquoting}.  The
program is traversed, binding action, unit, and ingredient declarations into
appropriate environments, and substituting where necessary. For example, when
a top-level ingredient declaration for \texttt{Dry Ingredients} is
encountered, it’s name is bound to it’s corresponding ingredient expression
and, when the name \texttt{Dry Ingredients} is encountered later in the
recipe, it will be substituted for the corresponding ingredient expression.
Finally, all top-level ingredient expressions, action bindings, and unit
bindings are bound to corresponding variables in Haskell to be used by the
programmer.  After the reification process, each ingredient expression bound
in Haskell will have all references to other ingredients declared in the
recipe substituted.

\section{Example Usage}\label{sec:exampleusage}
Among our primary motivations for the language was the desire for a platform
upon which to build cooking-related applications. To demonstrate the power and
usability of GastroGnome as a foundation for apps, we have implemented a few
of examples as a proof of concept. While not wrapped into polished apps, they
show that functionality can be quickly reached when the data representation is
abstracted into GastroGnome.

The first is a simple program which constructs a shopping list given a list of
ingredients already available and one or more recipes to buy for. The program
extracts the ingredients and quantities from the recipe and simply subtracts
the ingredients which are already owned. The code for this application was
relatively straightforward and implemented in 46 lines of Haskell.

It is widely accepted that one of the most difficult things to master as a
chef is getting multiple dishes to finish at the same time. The second
application aims to remedy this by converting a list of GastroGnome recipes
into a sequence of time-indexed actions. Given a desired meal time, following
the resulting steps at the computed time will ensure that the recipes finish
simultaneously. This can easily be converted into a notification-based
application for a phone or tablet which sits on the counter as the chef works,
ensuring steps are done in the optimal order and that all of the dishes are
done on time. This application was implemented in 80 lines of Haskell.

\section{Evaluation}
In order to evaluate the success of GastroGnome, we intend to compare recipes
written in natural language (as it would appear in a cookbook) with their
representations in GastroGnome. The recipes will be compared on the grounds of
length, readability, and accessibility to non-programmers. In our limited
testing, GastroGnome recipe translations are on average shorter than their
English counterparts. As both designers of the language and experienced
programmers, we acknowledge our own biases and can speak neither on the
readability of the language nor the ease of interpretation by a chef without
programming experience.

We also intend to assess the barrier to entry for the language by asking
non-experts to read a simple introduction document, then test their ability to
produce GastroGnome translations of real recipes taken from cookbooks. We hope
that even the most lay of people will exhibit the capacity to produce
consistent translations and that two subjects given the same task will produce
functionally (if not syntactically) identical GastroGnome programs.

\section{Future Work}
While we believe GastroGnome is already in a very usable state, there are a
number of language features we would like to see in the future. There are a
few noticeable gaps in GastroGnome, one of which is the handling of units.
Currently, units are essentially treated as string atoms. We would like to
support unit conversion and user-defined units in the future. Another issue is
the lack of representation of appliances and cooking tools such as a pans or
ovens. Currently these are implemented as ingredients and referenced in action
modifiers. We feel this could be remedied by adding a language construct for
such appliances. Similarly, we feel support for certain key modifiers such as
temperatures and cooking time could be better supported. We will also consider
the benefits of a stronger type system and type checking to aid in writing
recipes.

We hope also to integrate import and export of GastroGnome recipes to and from
files or structured data storage. The ability to keep large, organized
collections of recipes is a crucial function for recipes. One of the great
benefits to having a programming language for recipes is the potential for
code re-use. Recipes could reference components from other recipes (e.g.
define Blueberry Waffles as Waffle Batter plus Blueberries). This will require
increased support for namespace and scoping.

There are also a number of tools which might prove useful including a visual
recipe editor which can toggle between an ingredient tree and a sequential
step view of the recipe. We are also interested in the potential for a
GastroGnome to be compiled to run on cooking robots. While the realization of
an automated chef may not be imminent, we believe a standardized language
could be a large boon to the development of cooking robots, and that the
existence of such robots might, in turn, help drive the development of
GastroGnome.

In a similar vein, there is room for integration of natural language
processing to translate recipes between English (or other languages) and
GastroGnome. This, while perhaps just out of reach of current technologies,
could provide a much larger base of recipes for GastroGnome without additional
effort to write new recipes explicitly in the language.


\section{Conclusion}
The use of natural language recipes for cooking has a number of flaws, perhaps
the greatest of which is ambiguity. We introduced GastroGnome as a new
standardized representation for recipes. Gastrognome aims to remedy many of
the shortcomings of natural language recipes. We have also demonstrated the
ease of development of useful applications using GastroGnome as an easy recipe
representation.

\section{Related Work}
We researched existing work in this domain and found Cordon Bleu, a related
language \cite{cordonbleu}. It is more focused on the simulation of flavors
in recipes, and we found that its more imperative syntax, while suited to its
stated task, was less ideal for a language which is intended for chefs who
may have minimal programming experience. We found that it has a large number
of primitive operations built in, which are necessary for the domain, but we
decided in contrast to make our language more lightweight and allow the end
user to use actions as they desire, and defer any interpretation of the
significance of action names to the tools which consume recipes.

\section{Acknowledgements}
We would like to thank Kathleen Fisher for advice and feedback over the course
of both planning and development for this project.

\bibliography{references}
\bibliographystyle{abbrvnat}
\nocite{*}

\end{document}
